\documentclass[12pt, a4paper]{scrartcl}
%\usepackage{a4wide}					%Platzintensiver
\usepackage[ngerman]{babel}	%Deutsche Silbentrennung
\usepackage[utf8]{inputenc}		%Damit kann man auch Umlaute direkt eintippen
\usepackage{textcomp}
\usepackage{amsmath}
\usepackage{amssymb}
\usepackage{color}
\usepackage{amsbsy}
\usepackage{mdwlist}
\usepackage{amsfonts}
\usepackage{calc}          % Erweiterte Syntax für Zahlen- und Längenangaben.
\usepackage[reals]{layout} % zur Kontrolle des aktuellen Layouts
\usepackage{geometry}
\geometry{top=30mm,bottom=30mm, textheight=210mm,textwidth=160mm,heightrounded,
right=20mm ,left=40mm}
\clubpenalty = 10000		% verhindert Schusterjungen (Einzelzeile unten)
\widowpenalty = 10000 
\displaywidowpenalty = 10000	% verhindert "Hurenkinder" (Einzelzeile oben)
\usepackage{amsthm}
\usepackage{graphicx}	%including images
\usepackage{fancyhdr}	%for editiable headers and footers
\usepackage{setspace} 	%for 1,5times row space
\usepackage[colorlinks,pdfpagelabels,pdfstartview = FitH,bookmarksopen = true,bookmarksnumbered = true,linkcolor = black,plainpages = false,hypertexnames = false,citecolor = black, linkcolor=black, menucolor=black, pagecolor=black, urlcolor=black] {hyperref}	
\usepackage{multirow}	%linebreak in Table
\usepackage{enumerate}
\usepackage{float}
\usepackage{colortbl}	%colorTables
\usepackage{framed}   %begin{framed}
\usepackage{longtable} %begin{longtable}
\usepackage{makeidx}
\usepackage[square]{natbib}
\usepackage{epigraph}

\include{parts/include-syntax-php}
%	Eigenes PHP Syntax-HL
\usepackage{listings}
\usepackage{caption}
\usepackage{epigraph}
\usepackage[dvipsnames]{xcolor}
\lstset{%
basicstyle=\ttfamily\footnotesize\color{black},
commentstyle = \ttfamily\color{gray},
keywordstyle=\ttfamily\color{blue},
stringstyle=\color{red},
showspaces=false,               % show spaces adding particular underscores
showstringspaces=false, 
frame=single
}


\renewcommand{\baselinestretch}{1.5}
\newcommand{\ownTitle}{Praktikumsbericht}
\newcommand{\ownTitleZ}{Entwicklung eines Gehaltsspiegels für Mitteldeutschland}
\newcommand{\ownAutor}{Akos Toth, HTW Dresden}
%\renewcommand{\familydefault}{\sfdefault}



\usepackage{hyphenat}
\usepackage[T1]{fontenc}


\hyphenation{
Mehr-wert
}

\parskip 5pt           % sets spacing between paragraphs
\parindent 0pt	
%%%%%%%%%%%%%%%%%%%%%%%%%%%%%%%%%%
%% HEAD AND FOOTLINES 
%%%%%%%%%%%%%%%%%%%%%%%%%%%%%%%%%%
\pagestyle{fancy}
\renewcommand{\sectionmark}[1]{\markboth{\thesection\ #1}{}}
\renewcommand{\subsectionmark}[1]{\markright{\thesubsection\ #1}}
\lhead[ \leftmark   ]{\textbf{\ownTitle}}
\rhead[\textbf{\ownTitleZ}]{\leftmark}
\lfoot[\thepage    ]{\scriptsize \textcopyright2012, \ownAutor}
\cfoot[]{}
\rfoot[\scriptsize \textcopyright2011, \ownAutor]{\thepage}

%%%%%%%%%%%%%%%%%%%%%%%%%%%%%%%%%%%%%%%
%  Macros
%%%%%%%%%%%%%%%%%%%%%%%%%%%%%%%%%%%%%%%%
\renewcommand{\footrulewidth}{0.5pt}
\newcommand{\cbox}{\includegraphics[width=0.2cm]{material/cbox.png} \ }
\definecolor{Gray}{rgb}{0.85,0.85,0.90}
%\setcounter{tocdepth}{2}	%kleineres TOC
\newcommand{\arr}{$ \Longrightarrow$}

\makeindex 	%start Aufzeichnun eines Stichwortverzeichnisses

%%%%%%%%%%%%%%%%%%%%%%%%%%%%%%%
%% Metainfos
%%%%%%%%%%%%%%%%%%%%%%%%%%%%%%%


\title{\ownTitle}
\author{\ownAutor}
\pdfinfo{
 /Title  (\ownTitle)
 /Author (\ownAutor)
 /Subject (\ownTitleZ)
 /Keywords()
}
\date{\today}
\onehalfspacing
\newcommand{\fullref}[1]{\ref{#1} -- \glqq\,\textit{\nameref{#1}}\grqq}
\setlength{\epigraphwidth}{.8\textwidth}
%%%%%%%%%%%%%%%%%%%%%%%%%%%%%%%%%%%%%%%%%%
\begin{document}
 \thispagestyle{empty}
\begin{center}
\begin{tabular}{lcr}
 \Large{HTW Dresden} & \verb|       |& \multirow{3}{*}{\includegraphics[height=1.353cm]{material/htwlogo.jpg}} \\
 \Large{Fachbereich Informatik} &  & \\
%\ownTitleZ &  & \\
\end{tabular}\end{center}
\begin{center}


\end{center}
\begin{verbatim}


\end{verbatim}
\begin{center}
\textbf{\Huge{\ownTitle}}


\end{center}
\begin{verbatim}



\end{verbatim}
\begin{center}
\textbf{\LARGE{\ownTitleZ}}
\end{center}
\begin{verbatim}








\end{verbatim}
\begin{flushleft}
\begin{tabular}{lll}
\textbf{Autor:} & & Toth, Akos\\
& & \\
\textbf{Seminargruppe} & & 08/042/62\\
& & \\
& & \\
\textbf{Betreuender Professor:} & & Prof. Dr.-Ing. Wiedemann\\
& & \\
& & \\
\textbf{Datum:} & & \today\\

\end{tabular}

\end{flushleft}
 \tableofcontents		%Inhaltsverzeichnis
 \part{Einleitung}
Die pludoni GmbH ist ein junges Dresdner Startup, welches die Vernetzung kleiner und mittelständischer Unternehmen in Sachsen, Sachsen-Anhalt und Thüringen 
mittels Empfehlungscommunitys im Fokus hat. Die Firma besteht seit 2009 unter der Leitung von Dr. Jörg Klukas und ist im stetigen Wachstum. 
Der Student ist 2010 als Werkstudent in das Unternehmen gekommen. Seine Aufgaben sind die Weiter- und Neuentwicklung von Produkten für und um die Empfehlungscommunitys. 
So auch die Entwicklung des hier beschriebenen Gehaltsbenchmarks. 
\section{Ziel}
Dieser Praktikumsbericht wird die Entwicklung eines Gehaltsbenchmarks für Mitteldeutschland mit dem Fokus auf den Bereich Softwareprogrammierer, Softwareentwickler und Software-Architekt beschreiben. 
Es werden beispielhafte Auszüge des entwickelten Codes vorgestellt. Im Detail geht der Praktikant auf Technische Hintergründe und Problemstellungen bei der Entwicklung ein.
\section{Rahmen des Praktikums}
Das Praktikum war in der Zeit von März 2011 bis September 2011 in der pluoni GmbH zu absolvieren. 
In dieser Zeit sollte die Konzeption und technische Umsetzung des ``Gehaltsbenchmarks für Mitteldeutschland'' erfolgen und zum praktischen Einsatz kommen.

 \part{Analyse}
Die Grundidee und Inspiration für einen Gehaltsbenchmark bezogen auf den Raum Mitteldeutschland kommt aus der Fachzeitschrift c't\footnote{Computerfachzeitschrift von heise.de}. 
In diesem Magazin erscheint jährlich eine Gehaltsumfrage. Diese bezieht sich aber im Gegensatz zu der des Studenten entwickelten auf gesamt Deutschland und sogar auf Österreich und die Schweiz.
Die Ergebnisse dieser Umfrage sind nur wenig Aussagekräftig wenn man kleinere Regionen betrachten möchte. 
Der Grund dafür ist, dass in den Vergangenen Jahren nur ca. 4000\footnote{Gehaltsumfrage 2008: http://www.heise.de/jobs/artikel/c-t-Gehaltsumfrage-2008-791357.html} Beschäftigte aus der IT-Branche teilgenommen haben.
\section{Analyse des Marktes}
Eine Gehaltsumfrage zu diesem Speziellen Thema gab es bisher noch nicht. 
Der Mehrwert des Gehaltsspiegels besteht für Mitarbeiter im Bereich \glossary{human-resources-management} darin, 
dass nun eine Auswertung der aktuellen Gehaltslage in einer bestimmten Region möglich ist. 
Gerade Personalleiter und Recruiter aus kleinen und mittelständischen Unternehmen können mit diesem Werkzeug fundierte Daten beziehen und sich eventuell neu Orientieren. 
\section{Analyse der technischen Voraussetzungen}
Als zu verwendende Technologien wurden die folgenden Voraussetzungen bestimmt:
\begin{itemize}
 \item einfache Umsetzung der Gestaltung von Fragebögen
 \item das System sollte mit einer MySQL Datenbank interagieren können
 \item es sollte eine einfache und klar strukturierte Oberfläche für den Administrator geben
 \item die Auswertung der Umfrage erfolgt in Form eines PDF's, welches per E-Mail verschickt werden kann
 \item die Anzeige der Auswertung ist für den Administrator Online einsehbar
 \item die Anzeige sollte mittels einer Javascript Bibliothek namens Flotr2\\footnote{http://humblesoftware.com/flotr2} zu realisieren
\end{itemize}

 \part{Entwurf}
Mit dem ``Gehaltsbenchmark für Mitteldeutschland`` sollen Personalleiter und Personalrecruiter einen Überblick über die aktuelle Gehaltssituation in Sachsen, Sachsen-Anhalt und Thüringen erhalten. 
Das Konzept sieht vor, dass Partner der Communitys ITsax.de und ITmitte.de nach abgeschlossener Teilnahme die Auswertung kostenfrei zur Verfügung gestellt bekommen. An der Studie darf kostenfrei teilgenommen werden. Für die Auswertung der Umfrage wird, wenn ein Teilnehmer kein Partner der eben erwähnten Communitys ist, ein Betrag von 990€ berechnet. 
Interessenten können eine allgemeine Auswertung auch ohne Teilnahme erhalten, wenn diese einen Betrag von 590€ entrichten. (Das bitte noch näher erläutern zweigeteilte Auswertung allgemein und Teilnehmerspezifisch)
\section{Entwurf der technischen Umsetzung}
Bei der Umsetzung des Projektes sollten verschieden Technologien zum Einsatz kommen die nachfolgend näher erläutert werden. Als Basis wurde Drupal 6 verwendet, da dieses System bereits bestand und es einfach war darauf aufzubauen. Ruby on Rails wurde als Backend verwendet, weil damit effektiv Daten aus relationalen Datenbanken verarbeitet werden können. Die pludoni GmbH verwendet zunehmend diese Framework. Auch aus dem Lernaspekt des Praxissemesters heraus entschied sich der Student für den Einsatz dieses Frameworks. Mittels Flotr2 ist es möglich große Datenmengen in Diagrammen darzustellen. Nach kurzer Einarbeitung in die Bibliothek konnte der Autor Ergebnisse erzielen und so die Entwicklung der neuen Software voran treiben. 
Als Programmiersprachen werden PHP, Ruby on Rails \cite{rails} und Javascript verwendet. 
\subsection{Drupal}
Drupal ist ein Content Management System. Es ermöglicht dem Nutzer durch ein eingebautes und leicht erweiterbares Menüsystem unterschiedliche Module zu aktivieren und zu verwenden. 
Der Vorteil besteht darin, dass der Administrator mittels Programmschnittstelle ohne Probleme neue Module hinzufügen kann ohne selbst Programmieren zu müssen. 
Diese Module werden von einer großen Community, die hinter dem Content Management System ``Drupal`` steht, entwickelt und veröffentlicht. %TODO Quelle auf Mitgliederzahl/Plugin/
\subsection{PHP}
``PHP ist eine sehr weit verbreitete Scriptsprache die speziell auf Webentwicklung zugeschnitten ist und in HTML eingebettet werden kann.''\footnote{http://www.php.net}
Drupal ist in PHP geschrieben und daher fiel die Auswahl des Studenten beim anpassen der Funktionalitäten auf diese Scriptsprache. Mit PHP kann der Student die vom Unternehmen gesetzten Anforderungen umsetzen und die zu verwendenden Module einfach und schnell anpassen.
\subsection{Ruby on Rails}
Für die Verarbeitung der Daten aus der Datenbank wurde Ruby on Rails verwendet. 
Dieses Framework nutzt Ruby als Sprache. Es erfuhr in den letzten Jahren sehr viel Aufmerksamkeit und setzte einen für damals revolutionären Standard im Bereich der Webanwendungsentwicklung.\footnote{Zitat aus der Quelle von Stefan}
\subsection{Javascript}
Nach ausgiebigen Recherchen im Internet hat sich der Autor für die Verwendung der Javascriptbibliothek Flotr2 entschieden. Der Grund dafür lag darin, dass es erstens gut Dokumentiert ist, zweitens sehr ansehnlich und drittens Opensource ist. Die grafische Darstellung der Ergebnisse haben bei der Entscheidung eine große Rolle gespielt.
{Beispielbild von Pie-Chart von Flotr2}
% TODO Entweder Technologien oder EInsatz beschreiben? Oder beides
\section{Datensicherheit}
Bei der Planung wurde im Vorfeld großes Augenmerk auf die Datensicherheit gelegt und somit als wichtiger Punkt in die Entwicklung einbezogen. Der Grund für einen solchen Punkt ist der, dass mit sensiblen Kundendaten verfahren wird und keiner der Kunden durch eventuelle Datenlücken oder ähnliches diskreditiert werden darf.
Ein Beispiel für solche Daten wird der Student nachfolgend beschreiben: 
Teilnehmer A aus Stadt X nimmt als Einziger aus Stadt X bei der Umfrage teil. Teilnehmer B kennt Teilnehmer A und weiß von ihm, dass er an dieser Umfrage teilnimmt. Nun kann Teilnehmer B Gehaltspannen von Teilnehmer A ablesen, weil die Auswertungen St\"adte basiert sind. Zum Schutz dieser Daten sollte ein Algorithmus implementiert werden, der genau diese Art von Offenlegung von Teilnehmern unterbinden muss. Dieser pr\"uft zuerst wie viele Teilnehmer einer Stadt vorhanden sind. Bei weniger als 3 Teilnehmern wird der Standort in der Auswertung nicht angezeigt. Wenn mindestens drei Teilnehmer aus einer Stadt kommen wird als n\"achstes gepr\"uft, ob insgesamt 15 abgegebene Antworten einer Frage vorhanden sind. Ist dies nicht der Fall wird der Standort auch nicht angezeigt. Mittels dieses Algorithmus' wird verhindert, dass Daten von Teilnehmern offengelegt werden.
 \part{Implementierung}
\begin{itemize}
 \item Railscode erkl\"aren berechnungen etc.
 \item Auszug aus Code der Berechnungen
 \item Aufbau PDF womit erstellt
 \item Adminansicht kurz beschreiben das es das gibt
 \item 
\end{itemize}


 \part{Ausblick}
Was wollen wir zukünftig mi dem Gehaltsbenchmark machen?
 \appendix
%\section{}
\listoffigures
%\listoftables
\cite{*} 
\bibliographystyle{dinat}
\bibliography{litverz}
\newpage

\section*{Selbstständigkeitserklärung}
   Hiermit erkläre ich, dass ich die vorliegende Arbeit
   selbstständig, unter Angabe aller Zitate und nur unter
   Verwendung der angegebenen Literatur und Hilfsmittel
   angefertigt habe. \\[2ex]
   Dresden, den \today \\[6ex]
   \begin{flushleft}
       \newlength\us
       \settowidth{\us}{-\ownAutor-}
       \begin{tabular}{p{\us}}\hline
           \centering\footnotesize \ownAutor
       \end{tabular}
   \end{flushleft}

\end{document}

