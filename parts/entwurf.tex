\part{Entwurf}
Mit dem ``Gehaltsbenchmark für Mitteldeutschland`` sollen Personalleiter und Personalrecruiter einen Überblick über die aktuelle Gehaltssituation in Sachsen, Sachsen-Anhalt und Thüringen erhalten. 
Das Konzept sieht vor, dass Partner der Communitys ITsax.de und ITmitte.de nach abgeschlossener Teilnahme die Auswertung kostenfrei zur Verfügung gestellt bekommen. An der Studie darf kostenfrei teilgenommen werden. Für die Auswertung der Umfrage wird, wenn ein Teilnehmer kein Partner der eben erwähnten Communitys ist, ein Betrag von 990€ berechnet. 
Interessenten können eine allgemeine Auswertung auch ohne Teilnahme erhalten, wenn diese einen Betrag von 590€ entrichten.
\section{Entwurf der technischen Umsetzung}
Bei der Umsetzung des Projektes sollten verschieden Technologien zum Einsatz kommen die nachfolgend näher erläutert werden. Als Basis wurde Drupal 6 verwendet, da dieses System bereits bestand und es einfach war darauf aufzubauen.
Als Programmiersprachen werden PHP, Ruby on Rails und Javascript verwendet. 
\subsection{Drupal}
Drupal ist ein \glossary{cms}. Es ermöglicht dem Nutzer durch ein eingebautes und leicht erweiterbares Menüsystem unterschiedliche Module zu aktivieren und zu verwenden. 
Es ist ein Modul basiertes System. Der Vorteil besteht darin, dass der Administrator ohne Probleme neue Module hinzufügen kann ohne selbst Programmieren zu müssen. 
Diese Module werden von einer großen Community, die hinter dem CMS ``Drupal``\footnote{http://drupal.org/} steht, entwickelt und veröffentlicht.
\subsection{PHP}
``PHP ist eine sehr weit verbreitete Scriptsprache die speziell auf Webentwicklung zugeschnitten ist und in HTML eingebettet werden kann.''\footnote{http://php.net}
Drupal ist in PHP geschrieben und daher fiel die Auswahl des Studenten beim anpassen der Funktionalitäten auf diese Scriptsprache. Mit PHP kann der Student die vom Unternehmen gesetzten Anforderungen umsetzen und die zu verwendenden Module einfach und schnell anpassen.
\subsection{Ruby on Rails}
Für die Verarbeitung der Daten aus der Datenbank wurde Ruby on Rails verwendet. 
Dieses Framework nutzt Ruby als Sprache. Es erfuhr in den letzten Jahren sehr viel Aufmerksamkeit und setzte einen für damals revolutionären Standard im Bereich der Webanwendungsentwicklung. 
Auch aus dem Lernaspekt des Praxissemesters heraus entschied sich der Student für den Einsatz dieses Frameworks.
\subsection{Javascript}
Nach ausgiebigen Recherchen im Internet hat sich der Autor für die Verwendung der Javascriptbibliothek Flotr2\footnote{http://humblesoftware.com/flotr2/} entschieden. Der Grund dafür lag darin, dass es erstens Opensource ist, zweitens gut Dokumentiert und drittens sehr ansehnlich ist. 
Nach kurzer Einarbeitung in die Bibliothek konnte der Student Ergebnisse erzielen und so die Entwicklung der neuen Software voran treiben. Die grafische Darstellung der Ergebnisse haben bei der Entscheidung eine große Rolle gespielt.
{Beispielbild von Pie-Chart von Flotr2}
\section{Datensicherheit}
Bei der Planung wurde im Vorfeld großes Augenmerk auf die Datensicherheit gelegt und somit als wichtiger Punkt in die Entwicklung einbezogen. Der Grund für einen solchen Punkt ist der, dass mit sensiblen Kundendaten verfahren wird und keiner der Kunden durch eventuelle Datenlücken oder ähnlichem Diskreditiert werden darf.
Ein Beispiel für solche Daten wird der Student nachfolgen Beschreiben: 
Teilnehmer A aus Stadt X nimmt als einziger aus Stadt X bei der Umfrage teil. Teilnehmer B kennt Teilnehmer A und wei\ss{} von ihm, dass er an dieser Umfrage teilnimmt. Nun kann Teilnehmer B Gehaltspannen von Teilnehmer A ablesen, weil die Auswertungen St\"adte basiert sind. Zum Schutz dieser Daten wurde ein sollte ein Algorithmus implementiert werden, der genau diese Art von Diskredition von Teilnehmern unterbinden muss. Dieser Pr\"uft zuerst wie viele Teilnehmer einer Stadt vorhanden sind. Bei weniger als 3 Teilnehmern wird der Standort in der Auswertung nicht angezeigt. Wenn mindestens drei Teilnehmer aus einer Stadt kommen wird als n\"achstes gepr\"uft ob insgesamt 15 abgegebene Antworten einer Frage vorhanden sind. Ist dies nicht der Fall wird der Standort auch nicht angezeigt. Mittels dieses Algorithmus' wird die Integrität der Daten gew\"ahrleistet.