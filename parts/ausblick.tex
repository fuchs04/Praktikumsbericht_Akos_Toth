\part{Ausblick}
\section{Empfehlungen zur weiteren Verwendung}
\subsection{technische Weiterverwendung}
\label{sec:ausblick_technik}
Der ``Gehaltsbenchmark für Mitteldeutschland'' kann sehr leicht für weitere Umfragen verwendet werden. Es muss für zukünftige Umfragen lediglich der Fragebogen und der Einleitungstext geändert werden.
Der technische Hintergrund kann beibehalten werden, da sich die Berechnungen im Hintergrund nicht ändern müssten. Der Student hat durch seine Arbeit mit Drupal und PHP in Verbindung mit dem Framework Ruby on Rails festgestellt, dass es praktikabler wäre, das Modul Webform für Drupal in das Framework zu übertragen, um eine einheitliche Entwicklungsgrundlage zu erhalten. Des weiteren könnte eine Art Baukasten für Fragebögen entwickelt werden, in dem nur das Thema und die Fragen erfasst werden und der Fragebogen automatisch generiert wird. 
\subsection{wirtschaftliche Weiterverwendung}
Aus dem wirtschaftlichen Blickwinkel könnte das Produkt in Newslettern und auf den Communityportalen beworben werden, um potentielle Abnehmer für weitere Studien zu gewinnen. Der Kostenfaktor wäre durch diese Art der Werbung nahezu Null, weil es bereits bestehende Communityportale gibt und die Entwicklung neuer "Werbeplattformen" entfällt. Das Prinzip der kostenlosen Teilnahme sollte nach Meinung des Studenten beibehalten werden, da sich daraus kostenfreier Mehrwert erzielen lassen kann. 
\section{Verbesserungen}
Durch die im Punkt \fullref{sec:ausblick_technik} erwähnte Übertragung des Moduls Webform in das Framework Ruby on Rails muss eine Überarbeitung des Speicherkonzeptes der Daten in der Datenbank erfolgen. Diese Überarbeitung sollte Daten wie zum Beispiel Gehälter verschlüsselt in der Datenbank ablegen, um die Sicherheit zu erhöhen. Eine weitere Verbesserung könnte nach Meinung des Studenten der automatische Versand der Umfrage sein. Dieser müsste nach Ablauf eines bestimmten Endtermins für die Umfrage erfolgen, ohne das Zutun von Mitarbeitern fordern zu müssen.
