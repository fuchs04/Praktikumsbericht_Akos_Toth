\part{Analyse}
Die Grundidee und Inspiration für einen Gehaltsbenchmark bezogen auf den Raum Mitteldeutschland kommt aus der Fachzeitschrift C'T \citep{ct}. 
In diesem Magazin erscheint jährlich eine Gehaltsumfrage. Diese bezieht sich aber im Gegensatz zu der des Studenten entwickelten auf das gesamte Bundesgebiet und sogar auf Österreich und die Schweiz. Einige Partner der Community ITsax.de regten bei einem Treffen zu diesem Projekt an, weil ihnen genau diese regionale Auswertung im IT-Bereich bisher fehlte.
Die Ergebnisse dieser Umfrage sind nur wenig aussagekräftig wenn man kleinere Regionen betrachten möchte. 
Der Grund dafür ist, dass in den vergangenen Jahren nur ca. 4000\footnote{Gehaltsumfrage 2008: http://www.heise.de/jobs/artikel/c-t-Gehaltsumfrage-2008-791357.html} Beschäftigte aus der IT-Branche teilgenommen haben \citep{ct}.
Das Projekt war eine Erweiterung des Webservices kanaleo.de \citep{kanaleo}. Dieser ist ein Werkzeug f\"ur Personalleiter und -recruiter, um herauszufinden über welchen Weg Bewerber und Mitarbeiter ins Unternehmen finden. Der Hintergrund ist, dass Unternehmen viel Geld f\"ur Stellenanzeigen ausgeben und keinen wirklichen Überblick darüber haben über welchen Kanal ihre Bewerber und Mitarbeiter ins Unternehmen gefunden haben. Mittels kanaleo.de werden Bewerber und Mitarbeiter befragt über welchen Kanal sie ins Unternehmen gekommen sind. Diese Daten werden ausgewertet und in Form von Säulen- und Tortendiagrammen dem Anwender zur Verfügung gestellt. 
\section{Analyse des Marktes}
Eine Gehaltsumfrage zu diesem speziellen Thema gab es bisher noch nicht. 
Der Mehrwert des Gehaltsspiegels besteht für Mitarbeiter im Bereich Human-Resources-Management darin, dass nun eine Auswertung der aktuellen Gehaltslage in einer bestimmten Region möglich ist. 
Gerade Personalleiter und Personalrecruiter aus kleinen und mittelständischen Unternehmen können mit diesem Werkzeug fundierte Daten beziehen und sich eventuell neu orientieren. 
\section{Analyse der technischen Voraussetzungen}
Als zu verwendende Technologien wurden die folgenden Voraussetzungen bestimmt:
\begin{itemize}
 \item einfache Umsetzung der Gestaltung von Fragebögen,
 \item das System sollte mit einer MySQL-Datenbank\footnote{http://www.mysql.de} interagieren können,
 \item es sollte eine einfache und klar strukturierte Oberfläche für den Administrator geben,
 \item die Auswertung der Umfrage erfolgt in Form eines PDFs\footnote{http://de.wikipedia.org/wiki/Portable\_Document\_Format}, welches per E-Mail verschickt werden kann,
 \item die Anzeige der Auswertung ist für den Administrator online einsehbar,
 \item die Anzeige sollte mittels einer Javascript-Bibliothek namens Flotr2 \citep{flotr} realisiert werden.
\end{itemize}